\documentclass[a4paper,
	DIN
]{scrlttr2}

% BEACHTE: Diese Datei ist Windows-1252/WinLatin1 (nicht ISO-8859-15) kodiert
%
% Build:
%
% kill %1 ; rm scrlttr2-work.pdf ; pdflatex scrlttr2-work.tex A B && open scrlttr2-work.pdf
                                      
% \usepackage[utf8]{inputenc}
\usepackage[latin1]{inputenc}

\KOMAoptions{foldmarks=bLMT,		% Faltmarken 1/3 und 1/2
	pagenumber=false, 				% keine Zeilennummern
	backaddress=true, 				% Rueckadresse im Sichtfenster
	refline=nodate,					% Betreffzeile ohne Datum
	fromalign=left, 				% Header = Adresse links, Logo rechts
	fontsize=10.0pt,
	fromrule=afteraddress,			% Linie unter der Adresse im Header
	fromlogo=true					% mit Logo
	}

\usepackage[scaled]{berasans}
\renewcommand*\familydefault{\sfdefault}  %% Only if the base font of the document is to be sans serif
\usepackage[T1]{fontenc}

\usepackage{german}					% deutsches Datumsformat & deutschen Rechtschreibung
\usepackage{hyperref}
                           
\usepackage{lmodern}				% richtige Schrift  
% \usepackage[demo]{graphicx}		% Option demo immer bei Minimalbeispielen und nie bei echten Dokumenten
\usepackage{graphicx}				% Option demo immer bei Minimalbeispielen und nie bei echten Dokumenten

\usepackage{csvsimple}

\usepackage{tabularx}
\newcolumntype{L}[1]{>{\raggedright\arraybackslash}p{#1}}	% linksb�ndig mit Breitenangabe
\newcolumntype{C}[1]{>{\centering\arraybackslash}p{#1}}		% zentriert mit Breitenangabe
\newcolumntype{R}[1]{>{\raggedleft\arraybackslash}p{#1}}	% rechtsb�ndig mit Breitenangabe

\newcommand{\ltab}{\raggedright\arraybackslash}				% Tabellenabschnitt linksb�ndig
\newcommand{\ctab}{\centering\arraybackslash}				% Tabellenabschnitt zentriert
\newcommand{\rtab}{\raggedleft\arraybackslash}				% Tabellenabschnitt rechtsb�ndig

\usepackage{pgf}
\pgfdeclaremask{mymask}{unterschrift-\gliederung}
\pgfdeclareimage[height=1cm,mask=mymask]{unterschrift}{unterschrift-\gliederung}
\pgfdeclareimage[height=14mm]{logo}{logo-\gliederung}


% \setkomavar{fromname}{\footnotesize{Bezeichnung und Anschrift der Partei:}}	% Alternative: \scriptsize

\setkomavar{subject}{}
\setkomavar{backaddress}{Piratenpartei \gliederungkurz \\ \gliederungstrasse \\ \gliederungort}

\newcommand*{\footbox}{
	\usekomafont{pagenumber}
	\centering
	\scriptsize{
		\begin{tabular}{L{7cm}R{8cm}}
		\hline
		Piratenpartei Deutschland \rule{0pt}{3ex} & Vorstandsvorsitzender: \gliederungeinsvor \\
		\gliederunglang & Schatzmeister: \gliederungschatzi \\
		\gliederungstrasse{} - \gliederungort & E-Mail: \gliederungkontakt \\
		\end{tabular}
	}
}

\firstfoot{\footbox}

\newcommand*{\headbox}
{
	\usekomafont{pagenumber}
	\parbox[b]{\linewidth}
	{
		\noindent
%		\fbox{
		\parbox[b]{0.5\linewidth}
		{
			\scriptsize{Bezeichnung und Anschrift der Partei:} \\ 
		\normalsize
			{
				Piratenpartei Deutschland \\ \gliederunglang \\ \gliederungstrasse \\ \gliederungort}
			}
%		}
		\hfill
%		\fbox{
		\pgfuseimage{logo}
%   	} 
	}
}
\setlength{\headsep}{2.5cm}

\firsthead{\headbox}


% Definition von Head/Footer f�r alle Seiten:
\usepackage{scrpage2}
\pagestyle{scrheadings}
\clearscrheadfoot
\ohead{\headbox}
\ofoot{\footbox}
