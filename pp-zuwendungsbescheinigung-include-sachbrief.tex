\newcommand\sachbrief[9]{

\begin{letter}{
#1 % Vorname Name
\\
#2		% Strasse
\\
#3		% PLZ Ort
\\                                    
#4		% Land
}

\opening{
\textbf{Best�tigung �ber Sachzuwendungen} \\

  \parbox[b]{\linewidth}{\noindent
  \fbox{
    \parbox[b]{0.3\linewidth}{
      \scriptsize{\textbf{Name und Anschrift des Zuwendenden:}}\\
      \normalsize{#1\\#2\\#3\\#4}
    }
  }
}

\vspace{3mm}
\small{im Sinne des � 34g, � 10b des Einkommensteuergesetzes an politische Parteien im Sinne des Parteiengesetzes}
}

\begin{center}
\begin{tabular}{|L{4cm}|L{8.5cm}|L{2.5cm}|}
\hline
\scriptsize{\textbf{Wer der Zuwendung - in Ziffern}}  &
\scriptsize{\textbf{- in Buchstaben -}} &
\scriptsize{\textbf{Tag der Zuwendung:}} \\
\ctab #5 & \ctab #6 & \ctab #7 \\
\hline
\end{tabular}

\vspace{2 mm}
\begin{tabular}{|L{15,85cm}|}
\hline
\scriptsize{\textbf{Genaue Bezeichnung der Sachzuwendung}} \\
\ctab #8 \\
\hline
\end{tabular}
\end{center}

\begin{tabular}{R{1cm}L{14cm}}
\fbox{\parbox[c][0.5em]{0.5em}{\hspace{0.5em}}} & \small{Die Sachzuwendung stammt nach den Angaben des Zuwendenden aus dem Betriebsverm�gen und ist mit dem Entnahmewert (ggf. mit dem niedrigeren gemeinen Wert) bewertet.} \vspace{1 mm} \\
\fbox{\parbox[c][0,5em]{0.5em}{\ifthenelse{#9=0}{\hspace{0.5em}}{X}}} & \small{Die Sachzuwendung stammt nach den Angaben des Zuwendenden aus dem Privatverm�gen.} \vspace{1 mm} \\
\fbox{\parbox[c][0,5em]{0.5em}{\hspace{0.5em}}} & \small{Der Zuwendende hat trotz Aufforderung keine Angaben zur Herkunft der Sachzuwendung gemacht.} \vspace{1 mm} \\
\fbox{\parbox[c][0,5em]{0.5em}{\ifthenelse{#9=0}{\hspace{0.5em}}{X}}} & \small{Geeignete Unterlagen, die zur Wertermittlung gedient haben, z.B. Rechnung, Gutachten, liegen vor.}
\end{tabular}

\vspace{5 mm}
%\small{
Es wird best�tigt, dass diese Zuwendung ausschlie�lich f�r die satzungsgem��en Zwecke verwendet wird.
%}

% \newcommand{\bedingung}{1}
% \ifx\gliederung\undefined GLIEDERUNG UNDEF \else HAB GLIEDERUNG \fi
% \ifx\bescheinigung\undefined BESCHEINIGUNG UNDEF \else HAB BESCHEINIGUNG \fi

\vspace{10 mm}
% \hspace{6cm}\includegraphics[height=1cm]{unterschrift.eps}

%\hspace{7.5cm}
%\ifthenelse{\unterschrift=1}{
%	\pgfuseimage{unterschrift}
%}{}

\begin{center}
\begin{tabular}{p{7cm}p{7cm}}
\bescheinigungort{}, \bescheinigungdatum & 
\ifx \bescheinigungausstellername \undefined \gliederungschatzi{}, Schatzmeister \else i.A. \bescheinigungausstellername, \bescheinigungausstellerfunktion \fi \\
\hline
\scriptsize{(Ort, Datum)} & \scriptsize{(Unterschrift, Funktion)}  \\
\end{tabular}
\end{center}

\vspace{5 mm}
\vspace{5 mm}
%\ifthenelse{\unterschrift=1}{
%	\textbf{Hinweise:}
%}{
	\textbf{Hinweis:}
%}

Wer vors�tzlich oder grob fahrl�ssig eine unrichtige Zuwendungsbest�tigung erstellt oder wer veranlasst, dass Zuwendungen nicht zu den in der Zuwendungsbest�tigung angegebenen steuerbeg�nstigten Zwecken verwendet werden, haftet f�r die entgangene Steuer (� 34g Satz 3, � 10b Abs. 4 EStG). 

%\ifthenelse{\unterschrift=1}{
%\vspace{5 mm}
%
%Diese Bescheinigung ist maschinell erstellt und ben�tigt gem�� EStR R 10b.1 Absatz 4 keine handschriftliche Unterschrift. Das Verfahren wurde am \unterschriftdatummeldung{} gegen�ber dem Finanzamt \unterschriftfinanzamt f�r die Piratenpartei Deutschland, \gliederunglang{}, Steuernummer \unterschriftsteuernummer, angezeigt.
%}{}

\end{letter}
}
