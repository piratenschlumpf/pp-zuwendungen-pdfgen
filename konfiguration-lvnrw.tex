% Info ueber die Gliederung
% Gliederungsname
\newcommand{\gliederunglang}{Landesverband Nordrhein-Westfalen}
\newcommand{\gliederungkurz}{LV NRW}

% Gliederungsanschrift
\newcommand{\gliederungstrasse}{Postfach 103041}
\newcommand{\gliederungort}{44030 Dortmund}

% 1.Vorsitz + Schatzmeisterei + Kontakt Email Adresse
\newcommand{\gliederungeinsvor}{Patrick Schiff}
\newcommand{\gliederungschatzi}{Volker Pulbert}
\newcommand{\gliederungkontakt}{schatzmeister@piratenpartei-nrw.de}

% Ort und Datum der Ausstellung der Bescheinigung
\newcommand{\bescheinigungort}{Neuheim}
\newcommand{\bescheinigungdatum}{19. Mai 2014}

% Falls \bescheinigungausstellername definiert ist dann wird dieses statt
% des Schatzmeisters angedruckt. Relevant z.B. fuer den Bund wo dies ggf
% zur Anwendung kommt. Bleibt es auskommentiert (mit % in erster Zeile)
% dann ist es undefiniert.
%\newcommand{\bescheinigungausstellername}{Irmchen}
%\newcommand{\bescheinigungausstellerfunktion}{Beauftragte}

% Wird hier bei \unterschrift eine 1 eingetragen, so wird eine eingescannte
% Unterschrift gem�� EStR R 10b.1 Absatz 4 sowie ein Hinweistext eingedruckt. 
% ! Nicht bei Sachspenden moeglich, die muessen manuell unterschrieben werden !
% Scan (NOETIG!): unterschrift-[GLIEDERUNG].png z.B. unterschrift-lvnrw.png
% Weiterhin notwendig: Das Verfahren muss dem Finanzamt angezeigt werden,
%                      die Steuernummer der Gliederung wird eingedruckt.
% D.h. auch \unterschriftdatummeldung, \unterschriftfinanzamt und 
% \unterschriftsteuernummer m�ssen korrekt angegeben sein.
\newcommand{\unterschrift}{1}
\newcommand{\unterschriftdatummeldung}{30.1.2014}
\newcommand{\unterschriftfinanzamt}{Dortmund-Ost}
\newcommand{\unterschriftsteuernummer}{317/5960/0912}
